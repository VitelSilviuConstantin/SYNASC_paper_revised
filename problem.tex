\section{Problem description}

The key points of the VBA-based malicious software are the accessibility and the popularity of Office Products. Being written in Visual Basic, macros has access to the majority of the functions implemented in Windows API. This, along with the mass usage of Office products, both in business and non-business fields, makes it one of the favourite tools for malware creators. According to a statistic made by Spiceworks\footnote{https://community.spiceworks.com/software/articles/2873-data-snapshot-the-state-of-productivity-suites-in-the-workplace}, nearly 82\% of companies use programs from the Microsoft Office package. Knowing this, it is easier for an attacker to build a malicious program which will be granted to run, with a certain probability, on the target machine.
	A common method of attack is based on the previously mentioned phishing attempts. The victim receives an email with a malicious attachment consisting of an Office document, claiming to be a legitimate file such as an invoice for a purchase or a legal document. Upon opening it, the user is prompted with a security warning stating that the macros have been disabled. In order to convince the target to enable the execution of the malicious code, the attackers add a message in the document which states that macros have to be enabled in order to access its content. This message represents the social engineering component of the attack. If the user is tricked into enabling the malicious macros, these will perform unwanted actions, such as encrypting a user's data, downloading and executing other malicious files, steal data, etc.
\par
Malicious macros are used both as a delivery method and as a standalone malicious entity. If, in the past, the main target of a malicious document was to spread itself across the network, nowadays the potential offered by those products made the malware creators to rethink their usage. It is more common to see a document whose target is to steal sensitive information from companies or to deliver a destructive component, for example, a ransomware.
\par
Considering the above observation, we developed a solution to detect malicious macros inside an office file and keep the rate of false positives as low as possible. Our system is based on a derived version of the perceptron, OSC\cite{OSC}. The system extracts the macros within a file, applies a lexical processing, extracts the sequence of features and feeds them to the machine learning algorithm. In this way, we provide a complete system to detect if an Office file is indeed malicious or not.
\par
Moreover, taking into account that the OSC is keeping the rate of false positives for its models at a small percentage, we can easily use and integrate them into an AV solution. Having a small amount of false positives is mandatory due to the fact that Office products are used in critical business infrastructures.
