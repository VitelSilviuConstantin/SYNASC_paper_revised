\section{Introduction}
In recent years, the attacks based on malicious scripts (VBA) from Microsoft Office documents have been growing in popularity and so has the necessity of detecting them. While the VBA language was originally developed as a powerful scripting language to help users automate tasks and create macro-driven applications, it became a prevalent infection vector due to its extended capabilities, such as accessing the native Windows system calls. As a response to the growing threat of malicious macros, Microsoft has implemented several security features to prevent the execution of unwanted code. Nowadays, as a countermeasure, the attackers distribute the malicious documents through spear-phishing\footnote{Spear-phishing is a phishing attempt targeted to certain individuals or companies. Phising is a fraudulent practice that aims to obtain sensitive information from users (such as passwords, credit card information, etc.) or trick them in performing certain actions  by concealing this practice under the impression of communicating with a trustworthy entity.} emails and use social engineering techniques in order to trick the victim into enabling the execution of the malicious macros.
\par
This paper aims to explain the current state of detections regarding this type of malicious files and analyzes the possibility of improving existing macro detections via machine learning techniques. We will consider a lightweight approach of this idea due to the limitations of it being used in an anti-malware solution where performance is a highly important requirement. Our solution is based on a derived version of the Perceptron algorithm which builds a detection model focused on the properties of the macro code extracted from the VBA project of Microsoft Office files.